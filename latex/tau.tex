\documentclass[a4paper,12pt]{article}
\usepackage{placeins}
\ifx\pdftexversion\undefined
  \usepackage[dvips]{graphicx}
\else
  \usepackage[pdftex]{graphicx}
\fi
\usepackage{amssymb,amsfonts,amstext,amsmath}
\usepackage{color}
\usepackage{setspace}                % Produzir espaçamento entre linhas
\usepackage{latexsym}
\usepackage[portuguese,brazilian]{babel}
\usepackage[utf8x]{inputenc}        % Para poder digitar os acentos da maneira usual:
\usepackage{aecompl}
\usepackage{graphicx}                % incluir figuras .bmp
\usepackage{wrapfig}
\usepackage{subfig}
\usepackage{calc}
\usepackage{mathrsfs}                % Fonte da densidade Hamiltoniana $\mathscr{H}$ e lagrengeana $\mathscr{L}$
\usepackage{dsfont}                  % Fontes para cojuntos:Ex. naturais -\usepackage{rotating}
\usepackage{float}
\usepackage{textcomp}
\usepackage{ctable}
\usepackage[T1]{fontenc}
\usepackage[round,authoryear]{natbib}
\usepackage{geometry}
\usepackage{authblk}

\geometry{verbose,a4paper,tmargin=3cm,bmargin=2cm,lmargin=3cm,rmargin=2cm,headsep=5mm,footskip=0cm}
%Formatar notas de rodape
\usepackage[hang]{footmisc} 
\setlength{\footnotemargin}{1em}
\usepackage[colorlinks,linkcolor=blue,bookmarks=true,citecolor=blue]{hyperref} %cria os links nas referencias e citacoes!
% Controlar linhas orfas e viuvas
\clubpenalty=10000
\widowpenalty=10000
\displaywidowpenalty=10000
\newcommand{\meanL}[1]{\relax\ifmmode \langle #1 \rangle_L \else $\langle #1 \rangle_L$\xspace \fi}
\newcommand{\mean}[1]{\relax\ifmmode \langle #1 \rangle \else $\langle #1 \rangle$\xspace \fi}
%\title{xxx}

\newcommand{\tauV}{\tau_{\mathrm{\textsc{v}}}}
\newcommand{\tauVN}{\tau_{\mathrm{\textsc{v}}}^{\mathrm{\textsc{neb}}}}
\newcommand{\Halpha}{\ifmmode \mathrm{H}\alpha \else H$\alpha$\xspace \fi}
\newcommand{\WHa}{\ifmmode W_{\mathrm{H}\alpha} \else $W_{\mathrm{H}\alpha}$\xspace \fi}
\newcommand{\Hbeta}{\ifmmode \mathrm{H}\beta \else H$\beta$\xspace \fi}
\newcommand{\NII}{[N\thinspace\textsc{ii}] $\lambda 6584$\xspace} 
\newcommand{\nII}{\ifmmode [\mathrm{N\,\textsc{ii}}] \else [N\thinspace\textsc{ii}]\xspace \fi}
\newcommand{\WnII}{W_{\mathrm{[N\,\textsc{ii}]}}}
\newcommand{\OIII}{[O\thinspace\textsc{iii}] $\lambda 5007$\xspace}
\newcommand{\oIII}{\ifmmode [\mathrm{O\,\textsc{iii}}] \else [O\thinspace{\sc iii}]\xspace \fi}
\newcommand{\del}[2]{\frac{\partial #1}{\partial #2}}

\begin{document}

\section{Initial outtakes about emission lines measurements}

Rub\'en words:

\begin{verbatim}
Tal y como están hechos los ajustes ahora, todo los resultados se guardan. 
Pero seguro que hay resultados que no son buenos. Un primer check sería 
ver si "pos", la longitud de onda central, está cerca del valor esperado 
(en reposo). Hay algunas veces que no hay linea, por ejemplo de [OII], 
pero cerca hay algún glitch o cualquier otra cosa y el programa lo mide. 
Poniendo una tolerancia de, digamos 2 o 3 Angtroms, eliminaría las líneas 
que se desvian de la posición central "teórica" fuera de esa tolerancia. 
Otro método de limpieza quizás serían FHWM absurdamente grandes, o EW. 
Pero creo que uno muy fiable a priori es el de la posición central de la 
línea''
\end{verbatim}

This kind of filter was not applied to this results. Maybe this has to be the next thing to do and do not appear to be a heavy thing to code, given a central lambda table. 

Only zones with H$\alpha$, H$\beta$, [O\thinspace{\sc iii}] and [N\thinspace{\sc ii}] signal to noise ratio $\leq$ 3 are used.

\section{The morphologic types}
The morphologic types used are evaluated using \texttt{get\_morfologia()} by ES.Enrique:

\begin{verbatim}
[‘E0','E1','E2','E3','E4','E5','E6','E7','S0','S0a','Sa','Sab','Sb',
[   0,   1,   2,   3,   4,   5,   6,   7,   8,  8.5,   9,  9.5,  10,  
'Sbc','Sc','Scd','Sd','Sdm','Sm','Ir']
 10.5,  11, 11.5,  12, 12.5,  13,  14]
\end{verbatim}

\section{Age range}
The ages used to stellar $\tau_V^{\star}$, SFR${}^\star$ and $\Sigma^\star$ calculations are: \

$\left( 10.01 , 25.2 , 63.2, 100.1 , 158.6 , 199.6 , 1400.2 \right) * 10^6$ years

The smallest light fraction (in the $flag_t$-ageMax age range) deemed to be OK for our stats is 0.05. Also the minimum $\tau_V^\star$ to be taken seriously: 0.05.

\section{Calculations}

\textbf{ALL THE CALCULATIONS ARE BY ZONE!! ZERO calculations are made by radius whatsoever.}
\\

$L_\lambda$ and $\epsilon (L_\lambda)$:
\begin{eqnarray}
	L_\lambda &=& 4 \pi d^2 F_\lambda \\
	\epsilon (L_\lambda) &=& 4 \pi d^2 \epsilon (F_\lambda)
\end{eqnarray}
\\

$L_{\Halpha}^{int}$ and $\tauVN$: 
\begin{eqnarray}
	L_\lambda^{obs} &=& L_\lambda^{int} e^{-\tau_\lambda} \\
	L_\lambda^{obs} &=& L_\lambda^{int} e^{-(\frac{\tau_\lambda}{\tauV}) \tauV} \\
	\frac{\tau_\lambda}{\tauV} &=& q_\lambda \\
	L_\lambda^{obs} &=& L_\lambda^{int} e^{-q_\lambda \tauV} \\
	\frac{L_\lambda^{obs}}{L_{\lambda^\prime}^{obs}} &=& \
\frac{L_\lambda^{int} e^{-q_\lambda \tauV}}{L_{\lambda^\prime}^{int} e^{-q_{\lambda^\prime} \tauV}} \\
	\ln \left(\frac{L_\lambda^{obs}}{L_{\lambda^\prime}^{obs}}\right) &=& \
\tauV (q_{\lambda^\prime} - q_\lambda) \ln \left(\frac{L_\lambda^{int}}{L_{\lambda^\prime}^{int}}\right) \\
	\tauV &=& \frac{1}{(q_{\lambda^\prime} - q_\lambda)} \left[\ln \ 
\left(\frac{L_\lambda^{obs}}{L_{\lambda^\prime}^{obs}}\right) - \
\ln \left(\frac{L_\lambda^{int}}{L_{\lambda^\prime}^{int}}\right)\right] \\
	\tauV &=& \frac{1}{(q_{\lambda^\prime} - q_\lambda)} \left[\ln \
\left(\frac{F_\lambda^{obs}}{F_{\lambda^\prime}^{obs}}\right) - \
\ln \left(\frac{F_\lambda^{int}}{F_{\lambda^\prime}^{int}}\right)\right] \\
	\tauVN &=& \frac{1}{(q_{\Hbeta} - q_{\Halpha})} \ln \left( \frac{ F_{\Halpha}^{obs} / F_{\Hbeta}^{obs}}{F_{\Halpha}^{int} / F_{\Hbeta}^{int}} \right) \\
	L_{\Halpha}^{int} &=& L_{\Halpha}^{obs} e^{(q_{\Halpha} \tauVN)} 
\end{eqnarray}
\\

$\epsilon(L_{\Halpha}^{int})$ error propagation:
\begin{eqnarray}
	L_{\Halpha}^{int} &\equiv& L_{\Halpha}^{int}(L_{\Halpha}^{obs}, \tauVN) \
\equiv L_{\Halpha}^{int}(L_{\Halpha}^{obs}, L_{\Hbeta}^{obs}) \\
	\epsilon (L_{\Halpha}^{int}) &=& \sqrt{\left(\del{L_{\Halpha}^{int}}{L_{\Halpha}^{obs}}\right)^2 \
\epsilon (L_{\Halpha}^{obs})^2 + \left(\del{L_{\Halpha}^{int}}{L_{\Hbeta}^{obs}}\right)^2 \
\epsilon (L_{\Hbeta}^{obs})^2 } \\
	\del{L_{\Halpha}^{int}}{L_{\Halpha}^{obs}} &=& e^{(q_{\Halpha} \tauVN)} \\
	\del{L_{\Halpha}^{int}}{L_{\Hbeta}^{obs}} &=& \left(\del{L_{\Halpha}^{int}}{\tauVN}\right) \
\left(\del{\tauVN}{L_{\Hbeta}^{obs}}\right) \\
	\del{L_{\Halpha}^{int}}{\tauVN} &=& L_{\Halpha}^{obs} q_{\Halpha} e^{(q_{\Halpha} \tauVN)} \\
	\del{\tauVN}{L_{\Hbeta}^{obs}} &=& - \frac{1}{L_{\Hbeta}^{obs} (q_{\Hbeta} - q_{\Halpha})} \\
	\del{L_{\Halpha}^{int}}{L_{\Hbeta}^{obs}} &=& - \
\left(\frac{q_{\Halpha}}{q_{\Hbeta} - q_{\Halpha}}\right) \
\left(\frac{L_{\Halpha}^{obs}}{L_{\Hbeta}^{obs}}\right) \\	
	\epsilon (L_{\Halpha}^{int}) &=& e^{(q_{\Halpha} \tauVN)} \ 
\sqrt{\epsilon(L_{\Halpha}^{obs})^2 + \	
\left(\frac{q_{\Halpha}}{q_{\Hbeta} - q_{\Halpha}}\right)^2 \ 
\left(\frac{L_{\Halpha}^{obs}}{L_{\Hbeta}^{obs}}\right)^2 \
\epsilon(L_{\Hbeta}^{obs})^2}
\end{eqnarray}
\\

$\tauVN$ error propagation:
\begin{eqnarray}
	\tauVN &\equiv& \tauVN(L_{\Halpha}^{obs}, L_{\Hbeta}^{obs}) \\
	\epsilon (\tauVN) &=& \sqrt{\left(\del{\tauVN}{L_{\Halpha}^{obs}}\right)^2 \
\epsilon (L_{\Halpha}^{obs})^2 + \left(\del{\tauVN}{L_{\Hbeta}^{obs}}\right)^2 \
\epsilon (L_{\Hbeta}^{obs})^2 } \\
	\del{\tauVN}{L_{\Halpha}^{obs}} &=& \frac{1}{L_{\Halpha}^{obs} (q_{\Hbeta} - q_{\Halpha})} \\
	\del{\tauVN}{L_{\Hbeta}^{obs}} &=& - \frac{1}{L_{\Hbeta}^{obs} (q_{\Hbeta} - q_{\Halpha})} \\
	\epsilon (\tauVN) &=& \frac{1}{(q_{\Hbeta} - q_{\Halpha})} \
\sqrt{\left(\frac{\epsilon (L_{\Halpha}^{obs})}{L_{\Halpha}^{obs}}\right)^2 + \
\left(\frac{\epsilon (L_{\Hbeta}^{obs})}{L_{\Hbeta}^{obs}}\right)^2 }
\end{eqnarray}
\\

$F_{\Halpha}^{obs} / F_{\Hbeta}^{obs}$ error propagation:
\begin{eqnarray}
	F_{Balmer}^{obs} &=& \frac{F_{\Halpha}^{obs}}{F_{\Hbeta}^{obs}} \\
	F_{Balmer}^{obs} &\equiv& F_{Balmer}^{obs}(F_{\Halpha}^{obs}, F_{\Hbeta}^{obs}) \\
	\epsilon(F_{Balmer}^{obs}) &=& \sqrt{\left(\del{F_{Balmer}^{obs}}{F_{\Halpha}^{obs}}\right)^2 \
\epsilon(F_{\Halpha}^{obs})^2 + \left(\del{F_{Balmer}^{obs}}{F_{\Hbeta}^{obs}}\right)^2 \
\epsilon(F_{\Hbeta}^{obs})^2} \\
	\del{F_{Balmer}^{obs}}{F_{\Halpha}^{obs}} &=& \frac{1}{F_{\Hbeta}^{obs}} \\
	\del{F_{Balmer}^{obs}}{F_{\Hbeta}^{obs}} &=& - \frac{F_{\Halpha}^{obs}}{F_{\Hbeta}^{obs}{}^2} \\
	\epsilon(F_{Balmer}^{obs}) &=& \frac{1}{F_{\Hbeta}^{obs}} \sqrt{\epsilon(F_{\Halpha}^{obs})^2 + \
\left(\frac{F_{\Halpha}^{obs}}{F_{\Hbeta}^{obs}}\right)^2 \epsilon(F_{\Hbeta}^{obs})^2} 
\end{eqnarray}
\\

$O3N2$ ([O\thinspace{\sc iii}]/[N\thinspace{\sc ii}]) and $\log Z_{neb}$:
\begin{eqnarray}
	O3N2 &=& \frac{F_{\oIII}^{int}}{F_{\nII}^{int}} = \
\frac{F_{\oIII}^{obs}}{F_{\nII}^{obs}} e^{\tauVN (q_{\oIII} - q_{\nII})} \\
	\log Z_{neb} &=& - 0.14 - 0.25 \log O3N2 \\
\end{eqnarray}
\\

$O3N2$ and $\log Z_{neb}$ error propagation:
\begin{eqnarray}
	O3N2 &\equiv& O3N2(F_{\oIII}^{obs}, F_{\nII}^{obs}, F_{\Halpha}^{obs}, F_{\Hbeta}^{obs}) \\
	\del{O3N2}{F_{\oIII}^{obs}} &=& e^{\tauVN (q_{\oIII} - q_{\nII})}\left(\frac{1}{F_{\nII}^{obs}}\right) \\
	\del{O3N2}{F_{\nII}^{obs}} &=& - e^{\tauVN (q_{\oIII} - q_{\nII})}\left(\frac{F_{\oIII}^{obs}}{F_{\nII}^{obs}{}^2}\right) \\
	\del{O3N2}{F_{\Halpha}^{obs}} &=& e^{\tauVN (q_{\oIII} - q_{\nII})} \
\left(\frac{q_{\oIII} - q_{\nII}}{q_{\Hbeta} - q_{\Halpha}}\right) \
\left(\frac{F_{\oIII}^{obs}}{F_{\nII}^{obs} F_{\Halpha}^{obs}}\right) \\
	\del{O3N2}{F_{\Hbeta}^{obs}} &=& - e^{\tauVN (q_{\oIII} - q_{\nII})} \
\left(\frac{q_{\oIII} - q_{\nII}}{q_{\Hbeta} - q_{\Halpha}}\right) \
\left(\frac{F_{\oIII}^{obs}}{F_{\nII}^{obs} F_{\Hbeta}^{obs}}\right) \\
	\nonumber \epsilon(O3N2) &=& \frac{e^{\tauVN (q_{\oIII} - q_{\nII})}}{F_{\nII}^{obs}} \
\sqrt{ \epsilon(F_{\oIII}^{obs})^2 + \left(\frac{F_{\oIII}^{obs}}{F_{\nII}^{obs}}\right)^2 \epsilon(F_{\nII}^{obs})^2 + ... } \\
	&& \overline{... + \left(\frac{q_{\oIII} - q_{\nII}}{q_{\Hbeta} - q_{\Halpha}}\right)^2 \left[\left(\frac{F_{\oIII}^{obs}}{F_{\Halpha}^{obs}}\right)^2 + \left(\frac{F_{\oIII}^{obs}}{F_{\Hbeta}^{obs}}\right)^2 \right]} \\
	\epsilon(\log Z_{neb}) &=& \sqrt{\del{\log Z_{neb}}{O3N2}^2 \epsilon(O3N2)^2} \\
	\del{\log Z_{neb}}{O3N2} &=& \frac{0.25}{\ln(10)} \frac{1}{O3N2} \\
	&=& \frac{0.25}{\ln(10)} \left(\frac{\epsilon(O3N2)}{O3N2}\right)
\end{eqnarray}
\\

$SFR_{\Halpha}$ calibration using the solar value for BC03 + Padova1994 + Salpeter:
\begin{eqnarray}
	SFR_{\Halpha} &=& 3.17 \left(M_\odot yr^{-1}\right) \times \frac{L(\Halpha)}{10^8}
\end{eqnarray}
\\

$\Sigma_{SFR}$. $\Sigma_{gas}$ and $\log$DGR:
\begin{eqnarray}
	\Sigma_{SFR} &=& \frac{SFR_{\Halpha}}{A_{zone}} \\
	\Sigma_{gas}{}^{1.4} &=& \frac{10^6 \Sigma_{SFR}}{1.6 \times 10^{-4}} \\
	\log DGR &=& C + \log \left(\frac{\tau_V^{neb}}{\Sigma_{gas}}\right)
\end{eqnarray}

\end{document}