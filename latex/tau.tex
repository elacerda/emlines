\documentclass[a4paper,12pt]{article}
\usepackage{placeins}
\ifx\pdftexversion\undefined
  \usepackage[dvips]{graphicx}
\else
  \usepackage[pdftex]{graphicx}
\fi
\usepackage{amssymb,amsfonts,amstext,amsmath}
\usepackage{color}
\usepackage{setspace}                % Produzir espaçamento entre linhas
\usepackage{latexsym}
\usepackage[portuguese,brazilian]{babel}
\usepackage[utf8x]{inputenc}        % Para poder digitar os acentos da maneira usual:
\usepackage{aecompl}
\usepackage{graphicx}                % incluir figuras .bmp
\usepackage{wrapfig}
\usepackage{subfig}
\usepackage{calc}
\usepackage{mathrsfs}                % Fonte da densidade Hamiltoniana $\mathscr{H}$ e lagrengeana $\mathscr{L}$
\usepackage{dsfont}                  % Fontes para cojuntos:Ex. naturais -\usepackage{rotating}
\usepackage{float}
\usepackage{textcomp}
\usepackage{ctable}
\usepackage[T1]{fontenc}
\usepackage[round,authoryear]{natbib}
\usepackage{geometry}
\usepackage{authblk}

\geometry{verbose,a4paper,tmargin=3cm,bmargin=2cm,lmargin=3cm,rmargin=2cm,headsep=5mm,footskip=0cm}
%Formatar notas de rodape
\usepackage[hang]{footmisc} 
\setlength{\footnotemargin}{1em}
\usepackage[colorlinks,linkcolor=blue,bookmarks=true,citecolor=blue]{hyperref} %cria os links nas referencias e citacoes!
% Controlar linhas orfas e viuvas
\clubpenalty=10000
\widowpenalty=10000
\displaywidowpenalty=10000
\newcommand{\meanL}[1]{\relax\ifmmode \langle #1 \rangle_L \else $\langle #1 \rangle_L$\xspace \fi}
\newcommand{\mean}[1]{\relax\ifmmode \langle #1 \rangle \else $\langle #1 \rangle$\xspace \fi}
%\title{xxx}

\newcommand{\tauV}{\tau_{\mathrm{\textsc{v}}}}
\newcommand{\tauVN}{\tau_{\mathrm{\textsc{v}}}^{\mathrm{\textsc{neb}}}}
\newcommand{\Halpha}{\ifmmode \mathrm{H}\alpha \else H$\alpha$\xspace \fi}
\newcommand{\WHa}{\ifmmode W_{\mathrm{H}\alpha} \else $W_{\mathrm{H}\alpha}$\xspace \fi}
\newcommand{\Hbeta}{\ifmmode \mathrm{H}\beta \else H$\beta$\xspace \fi}
\newcommand{\NII}{[N\thinspace\textsc{ii}] $\lambda 6584$\xspace} 
\newcommand{\nII}{\ifmmode [\mathrm{N\,\textsc{ii}}] \else [N\thinspace\textsc{ii}]\xspace \fi}
\newcommand{\WnII}{W_{\mathrm{[N\,\textsc{ii}]}}}
\newcommand{\OIII}{[O\thinspace\textsc{iii}] $\lambda 5007$\xspace}
\newcommand{\oIII}{\ifmmode [\mathrm{O\,\textsc{iii}}] \else [O\thinspace{\sc iii}]\xspace \fi}
\newcommand{\del}[2]{\frac{\partial #1}{\partial #2}}

\begin{document}

Cálculo de $L_\lambda$ e do erro ($\epsilon (L_\lambda)$)

\begin{eqnarray}
	L_\lambda &=& 4 \pi d^2 F_\lambda \\
	\epsilon (L_\lambda) &=& 4 \pi d^2 \epsilon (F_\lambda)
\end{eqnarray}

Cálculo de $L_{\Halpha}^{int}$ e $\tauVN$:
 
\begin{eqnarray}
	L_\lambda^{obs} &=& L_\lambda^{int} e^{-\tau_\lambda} \\
	L_\lambda^{obs} &=& L_\lambda^{int} e^{-(\frac{\tau_\lambda}{\tauV}) \tauV} \\
	\frac{\tau_\lambda}{\tauV} &=& q_\lambda \\
	L_\lambda^{obs} &=& L_\lambda^{int} e^{-q_\lambda \tauV} \\
	\frac{L_\lambda^{obs}}{L_{\lambda^\prime}^{obs}} &=& \
\frac{L_\lambda^{int} e^{-q_\lambda \tauV}}{L_{\lambda^\prime}^{int} e^{-q_{\lambda^\prime} \tauV}} \\
	\ln \left(\frac{L_\lambda^{obs}}{L_{\lambda^\prime}^{obs}}\right) &=& \
\tauV (q_{\lambda^\prime} - q_\lambda) \ln \left(\frac{L_\lambda^{int}}{L_{\lambda^\prime}^{int}}\right) \\
	\tauV &=& \frac{1}{(q_{\lambda^\prime} - q_\lambda)} \left[\ln \ 
\left(\frac{L_\lambda^{obs}}{L_{\lambda^\prime}^{obs}}\right) - \
\ln \left(\frac{L_\lambda^{int}}{L_{\lambda^\prime}^{int}}\right)\right] \\
	\tauV &=& \frac{1}{(q_{\lambda^\prime} - q_\lambda)} \left[\ln \
\left(\frac{F_\lambda^{obs}}{F_{\lambda^\prime}^{obs}}\right) - \
\ln \left(\frac{F_\lambda^{int}}{F_{\lambda^\prime}^{int}}\right)\right] \\
	\tauVN &=& \frac{1}{(q_{\Hbeta} - q_{\Halpha})} \ln \left( \frac{ F_{\Halpha}^{obs} / F_{\Hbeta}^{obs}}{F_{\Halpha}^{int} / F_{\Hbeta}^{int}} \right) \\
	L_{\Halpha}^{int} &=& L_{\Halpha}^{obs} e^{(q_{\Halpha} \tauVN)} 
\end{eqnarray}

Propagação de erro no cálculo de $L_{\Halpha}^{int}$:

\begin{eqnarray}
	L_{\Halpha}^{int} &\equiv& L_{\Halpha}^{int}(L_{\Halpha}^{obs}, \tauVN) \
\equiv L_{\Halpha}^{int}(L_{\Halpha}^{obs}, L_{\Hbeta}^{obs}) \\
	\epsilon (L_{\Halpha}^{int}) &=& \sqrt{\left(\del{L_{\Halpha}^{int}}{L_{\Halpha}^{obs}}\right)^2 \
\epsilon (L_{\Halpha}^{obs})^2 + \left(\del{L_{\Halpha}^{int}}{L_{\Hbeta}^{obs}}\right)^2 \
\epsilon (L_{\Hbeta}^{obs})^2 } \\
	\del{L_{\Halpha}^{int}}{L_{\Halpha}^{obs}} &=& e^{(q_{\Halpha} \tauVN)} \\
	\del{L_{\Halpha}^{int}}{L_{\Hbeta}^{obs}} &=& \left(\del{L_{\Halpha}^{int}}{\tauVN}\right) \
\left(\del{\tauVN}{L_{\Hbeta}^{obs}}\right) \\
	\del{L_{\Halpha}^{int}}{\tauVN} &=& L_{\Halpha}^{obs} q_{\Halpha} e^{(q_{\Halpha} \tauVN)} \\
	\del{\tauVN}{L_{\Hbeta}^{obs}} &=& - \frac{1}{L_{\Hbeta}^{obs} (q_{\Hbeta} - q_{\Halpha})} \\
	\del{L_{\Halpha}^{int}}{L_{\Hbeta}^{obs}} &=& - \
\left(\frac{q_{\Halpha}}{q_{\Hbeta} - q_{\Halpha}}\right) \
\left(\frac{L_{\Halpha}^{obs}}{L_{\Hbeta}^{obs}}\right) \\	
	\epsilon (L_{\Halpha}^{int}) &=& e^{(q_{\Halpha} \tauVN)} \ 
\sqrt{\epsilon(L_{\Halpha}^{obs})^2 + \	
\left(\frac{q_{\Halpha}}{q_{\Hbeta} - q_{\Halpha}}\right)^2 \ 
\left(\frac{L_{\Halpha}^{obs}}{L_{\Hbeta}^{obs}}\right)^2 \
\epsilon(L_{\Hbeta}^{obs})^2}
\end{eqnarray}

Propagação de erro no cálculo de $\tauVN$:
\begin{eqnarray}
	\tauVN &\equiv& \tauVN(L_{\Halpha}^{obs}, L_{\Hbeta}^{obs}) \\
	\epsilon (\tauVN) &=& \sqrt{\left(\del{\tauVN}{L_{\Halpha}^{obs}}\right)^2 \
\epsilon (L_{\Halpha}^{obs})^2 + \left(\del{\tauVN}{L_{\Hbeta}^{obs}}\right)^2 \
\epsilon (L_{\Hbeta}^{obs})^2 } \\
	\del{\tauVN}{L_{\Halpha}^{obs}} &=& \frac{1}{L_{\Halpha}^{obs} (q_{\Hbeta} - q_{\Halpha})} \\
	\del{\tauVN}{L_{\Hbeta}^{obs}} &=& - \frac{1}{L_{\Hbeta}^{obs} (q_{\Hbeta} - q_{\Halpha})} \\
	\epsilon (\tauVN) &=& \frac{1}{(q_{\Hbeta} - q_{\Halpha})} \
\sqrt{\left(\frac{\epsilon (L_{\Halpha}^{obs})}{L_{\Halpha}^{obs}}\right)^2 + \
\left(\frac{\epsilon (L_{\Hbeta}^{obs})}{L_{\Hbeta}^{obs}}\right)^2 }
\end{eqnarray}

Propagação do erro no cálculo de $F_{\Halpha}^{obs} / F_{\Hbeta}^{obs}$:
\begin{eqnarray}
	F_{Balmer}^{obs} &=& \frac{F_{\Halpha}^{obs}}{F_{\Hbeta}^{obs}} \\
	F_{Balmer}^{obs} &\equiv& F_{Balmer}^{obs}(F_{\Halpha}^{obs}, F_{\Hbeta}^{obs}) \\
	\epsilon(F_{Balmer}^{obs}) &=& \sqrt{\left(\del{F_{Balmer}^{obs}}{F_{\Halpha}^{obs}}\right)^2 \
\epsilon(F_{\Halpha}^{obs})^2 + \left(\del{F_{Balmer}^{obs}}{F_{\Hbeta}^{obs}}\right)^2 \
\epsilon(F_{\Hbeta}^{obs})^2} \\
	\del{F_{Balmer}^{obs}}{F_{\Halpha}^{obs}} &=& \frac{1}{F_{\Hbeta}^{obs}} \\
	\del{F_{Balmer}^{obs}}{F_{\Hbeta}^{obs}} &=& - \frac{F_{\Halpha}^{obs}}{F_{\Hbeta}^{obs}{}^2} \\
	\epsilon(F_{Balmer}^{obs}) &=& \frac{1}{F_{\Hbeta}^{obs}} \sqrt{\epsilon(F_{\Halpha}^{obs})^2 + \
\left(\frac{F_{\Halpha}^{obs}}{F_{\Hbeta}^{obs}}\right)^2 \epsilon(F_{\Hbeta}^{obs})^2} 
\end{eqnarray}

Cálculo de $O_3N_2$ e $\log Z_{neb}$:
\begin{eqnarray}
	O_3N_2 &=& \frac{F_{\oIII}^{int}}{F_{\nII}^{int}} = \
\frac{F_{\oIII}^{obs}}{F_{\nII}^{obs}} e^{\tauVN (q_{\oIII} - q_{\nII})} \\
	\log Z_{neb} &=& - 0.14 - 0.25 \log O_3N_2 \\
\end{eqnarray}

Propagação no erro de $O_3N_2$ e $\log Z_{neb}$:
\begin{eqnarray}
	O_3N_2 &\equiv& O_3N_2(F_{\oIII}^{obs}, F_{\nII}^{obs}, F_{\Halpha}^{obs}, F_{\Hbeta}^{obs}) \\
	\del{O_3N_2}{F_{\oIII}^{obs}} &=& e^{\tauVN (q_{\oIII} - q_{\nII})}\left(\frac{1}{F_{\nII}^{obs}}\right) \\
	\del{O_3N_2}{F_{\nII}^{obs}} &=& - e^{\tauVN (q_{\oIII} - q_{\nII})}\left(\frac{F_{\oIII}^{obs}}{F_{\nII}^{obs}{}^2}\right) \\
	\del{O_3N_2}{F_{\Halpha}^{obs}} &=& e^{\tauVN (q_{\oIII} - q_{\nII})} \
\left(\frac{q_{\oIII} - q_{\nII}}{q_{\Hbeta} - q_{\Halpha}}\right) \
\left(\frac{F_{\oIII}^{obs}}{F_{\nII}^{obs} F_{\Halpha}^{obs}}\right) \\
	\del{O_3N_2}{F_{\Hbeta}^{obs}} &=& - e^{\tauVN (q_{\oIII} - q_{\nII})} \
\left(\frac{q_{\oIII} - q_{\nII}}{q_{\Hbeta} - q_{\Halpha}}\right) \
\left(\frac{F_{\oIII}^{obs}}{F_{\nII}^{obs} F_{\Hbeta}^{obs}}\right) \\
	\nonumber \epsilon(O_3N_2) &=& \frac{e^{\tauVN (q_{\oIII} - q_{\nII})}}{F_{\nII}^{obs}} \
\sqrt{ \epsilon(F_{\oIII}^{obs})^2 + \left(\frac{F_{\oIII}^{obs}}{F_{\nII}^{obs}}\right)^2 \epsilon(F_{\nII}^{obs})^2 + ... } \\
	&& \overline{... + \left(\frac{q_{\oIII} - q_{\nII}}{q_{\Hbeta} - q_{\Halpha}}\right)^2 \left[\left(\frac{F_{\oIII}^{obs}}{F_{\Halpha}^{obs}}\right)^2 + \left(\frac{F_{\oIII}^{obs}}{F_{\Hbeta}^{obs}}\right)^2 \right]} \\
	\epsilon(\log Z_{neb}) &=& \sqrt{\del{\log Z_{neb}}{O_3N_2}^2 \epsilon(O_3N_2)^2} \\
	\del{\log Z_{neb}}{O_3N_2} &=& \frac{0.25}{\ln(10)} \frac{1}{O_3N_2} \\
	&=& \frac{0.25}{\ln(10)} \left(\frac{\epsilon(O_3N_2)}{O_3N_2}\right)
\end{eqnarray}

\end{document}